\section{Geld}
Das Geld erleichtert die Zahlungsvorgänge bei Tauschgeschäften. Als Geld können verschiedene Zahlungsmittel dienen sofern, sie folgende Ansprüchen erfüllen.
\begin{itemize}
	\item Tauschmittel reduziert Transaktionskosten
	\item Wertaufbewahrungsmittel
	\item Masseinheit
\end{itemize}
\subsection{Wer schafft Geld}
\begin{itemize}
	\item Gold ist das anerkannte Zahlungsmittel im 15. Jahrhundert
	\item Besitzer hinterlegen Gold bei Goldschmieden. Diese stellen dafür Quittungen aus.
	\item Die Quittungen erhalten Geldcharakter (100\% Golddeckung)
	\item Goldschmiede erkennen dass über einen längeren Zeitraum nur ein Teil des hinterlegten Geldes abgeholt wird
	\item Goldschmiede beginnen überproportional viele Quittungen auszustellen
	\item Damit werden sie zu Bankiers. Keine 100\% Golddeckung sowie die Fähigkeit zur Rückzahlung basiert auf Vertrauensbasis (Zahlungsfähigkeit)
	\item Die Geschäftsbanken können heute Geld schaffen
	\item Um das Vertrauen in die Banken zu erhalten ist eine Regulierung notwendig
	\item Der Mindestreserve-Satz (RS) beschränkt Geldschöpfung
	\item Geldschöpfungsmultiplikator $GM= 1/RS$
\end{itemize}
\subsection{Geldmengen}
\begin{itemize}
	\item Notenbankgeldmenge M0
	\subitem Noten Girokonten der Geschäftsbanken
	\item Geldmenge M1
	\subitem M0 + Sichteinlagen (Zahlungszwecke)
	\item Geldmenge M2
	\subitem M1 + Spareinlagen
	\item Geldmenge M3
	\subitem M2 + Termineinlagen
\end{itemize}
\subsection{Instrumente der Geldpolitik}
\subsubsection{Offenmarktpolitik}
\begin{itemize}
	\item Kauf und Verkauf von Wertschriften (Aktiva)
	\subitem Nationalbank bezahlt Aktive mit frischen Geld
	\subitem Kauf von Aktiva ist ein Repogeschäft d.h. Der Verkäufer verpflichtet sich zum späteren Rückkauf dieser Aktiva
	\item Ausgabe von Schuldverschreibungen 
\end{itemize}
\subsubsection{Mindestreservepolitik}
\begin{itemize}
	\item Nationalbank kann den Mindestreservesatz festlegen
	\item Damit wird der Geldschöpfungsmultiplikator beeinflusst
\end{itemize}
\subsection{Geldpolitische Strategien}
\begin{itemize}
	\item Wechselkursziele
	\subitem Fixierung des Wechselkurs an internationale Leitwährung
	\subitem Gefahr eines Importes von Inflation
	\item Geldmengenziele
	\subitem Monetaristischer Ansatz $P \cdot Q = M \cdot V$
	\subitem funktioniert nur solange wie V konstant ist
	\item Inflationsziele
	\subitem Wahrung der Preisstabilität
\end{itemize}
\subsection{Schweizerische Nationalbank SNB}
\begin{itemize}
	\item Vorrangiges Ziel ist die Preisstabilität und erst anschliessend die Konjunktur
	\item Der Bundesrat hat keine Weisungsbefugnis gegenüber der SNB
\end{itemize}