\section{Die grosse Finanzkrise}
Nach 2001 verfolgte die USA eine sehr expansive Geldpolitik. Dadurch waren die Zinsen lange Zeit ungewöhnlich tief. Die tiefen Hypothekarzinsen verursachten ein rasches Ansteigen der Immobiliennachfrage und damit der Häuserpreise.
Gleichzeitig wurde ein innovatives Finanzinstrumentes kreiert –die ABS (asset-backed security) . Damit konnten neu Tausende von einzelnen Hypotheken gebündelt und in handelbare Wertpapiere verwandelt werden (Verbriefung). Die ABS wurden als risikolos (Triple-A) eingeschätzt und rentierten besser als vergleichbare Anlagen wie z.B. Staatsanleihen.
Damit floss viel Kapital in den US-amerikanischen Immobiliensektor, die Häuserpreise steigen unvermindert weiter und bildeten mit der Zeit eine eigentliche Finanzblase.
Schwächen in der Regulierung des US-amerikanischen Immobilienmarktes führten dazu, dass zunehmend auch an finanzschwache Haushalte (Subprime-Segment) grosszügige Immobilienkredite vergeben wurden.
Kreditgeber wie auch Kreditnehmer spekulierten auf zukünftige Wertsteigerungen der gekauften Immobilie. Mit diesen Wertsteigerungen sollten die anfallenden Zinszahlungen beglichen werden.\\
Mitte 2006 begannen die Häuserpreise in den USA landesweit drastisch zu sinken. Gründe dazu waren. 
\begin{itemize}
	\item Ab 2004 hob die amerikanische Zentralbank schrittweise die Zinsen wieder an. Die Hypothekarzinsen der Subprime-Kredite stiegen dadurch rasch an.
	\item Die Immobiliennachfrage ging wegen den höheren Hypothekarzinsen zurück. Zudem mussten erste «Subprime»-Kreditnehmer ihre Immobilien an die Banken zurückgeben, weil sie die Zinsen nicht mehr bedienen konnten.
	\item Mit der zurückgehenden Nachfrage und dem steigenden Angebot nach Immobilien kamen die Häuserpreise rasch unter Druck. Die Banken reagierten mit einer vorsichtigeren Vergabe von Krediten (die Nachfragereduktion verstärkt sich dadurch).
	\item Die Preise der ABS sackten dadurch ins Bodenlose ab.
\end{itemize}
Dadurch wurde das Vertrauen in das internationale Bankensystem erschüttert. Die Zinsen im Interbankengeschäft stiegen stark und es wurde kein Geld mehr untereinander verliehen. Die Notenbanken mussten einspringen. 