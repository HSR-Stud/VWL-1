\section{Staatsfinanzen}
Formen von Staatseinnahmen
\begin{itemize}
	\item Steuren
	\subitem Direkte Steuern
	\subitem Indirekte Steuern
	\subitem Gebühren
	\item Verschuldung
	\subitem Deckung von Investitionen
	\subitem Deckung des laufenden Staatsaufwandes
	\item Inflationssteuer
	\subitem Besteuerung der Geldhaltung durch Ausweitung der Geldmenge
\end{itemize}
\subsection{Staatsverschuldung}
\begin{itemize}
	\item Verschuldung im Inland führt zum Rückgang inländischer privater Investoren
	\subitem Erhöhung der staatlichen Kreditnachfrage auf dem inländischen Kapitalmarkt -> steigende Kreditzinsen -> crowding-out der Unternehmensinvestitionen
	\item Verschuldung im Ausland führt zum Rückgang von Exporten
	\subitem Erhöhung der staatlichen Kreditnachfrage auf dem ausländischen Kapitalmarkt in ausländischer Währung -> Umtausch in inländische Währung -> erhöhte Nachfrage nach inländischer Währung führt zu deren Aufwertung -> deshalb sinkende Exporte
	\item Staatsverschuldung ist attraktiver als Steuererhöhung
	\item Stimmentausch Parlament
	\item Trennung von Ausgabenbeschluss und Einnahmenentscheid
	\item Inflationssteuer: Einnahmen des Staates durch übermässige Geldschöpfung
	\item Explizite Staatsschuld: Schulden durch Kredite
	\item Implizite Staatsschuld: alle zukünftigen Versprechen (Renten, Versicherung etc.)
\end{itemize}
\begin{multicols}{2}
	\textbf{Vorteile}
	\begin{itemize}
		\item Finanzierung von langfristigen Investitionen
		\item Steuerglättung, jährliche Anpassung Steuersätze ist unmöglich
		\item Makroökonomische Stabiliserung, Ausgleich von konjunkturellen Schwankungen
	\end{itemize}
	\textbf{Nachteile}
	\begin{itemize}
		\item Verdrängung privater Investitionen
		\item Zunehmende Staatsschulden führen zu höheren Zinssätzen und damit Zinslast als Budegtposten
		\item Erhebung einer Inflationssteuer
	\end{itemize}
\end{multicols}
\subsection{Steuersystem der Schweiz}
\begin{itemize}
	\item Direkte Steuern ,2/3 der Einnahmen und indirekte Steuern (Mehrwertsteuer)
	\item Kantone und Gemeinde 2/3 der Staatseinnahmen
	\item Steuerwettbewerb
	\item Finanzausgleich
	\item Progressives Steuersystem, je mehr man verdient desto mehr zahlt man. In der Schweiz zahlt circa 10\% der Bevölkerung rund 50\% der Bundessteuern. 
	\item Schuldenbremse: Der Bund behält seine Ausgaben und Einnahmen im Gleichgewicht in Abhängigkeit der Konjunktur. Somit soll in Boom ein Überschuss erzielt werden um den Verlust in der Rezession aufzufangen. 
	\item Die Schuldenbremse ist in der Bundesverfassung festgeschrieben
	\item Die Schuldenbremse gibt es auch auf Kantonsebene
\end{itemize}
\clearpage
\pagebreak